\chapter{Lab Rules and Policies}

The purpose of this laboratory is to provide a hands on experience for students in conjunction with the recorded lectures and problem sessions. While solving nodal equations and drawing circuits has a lot of utility, there is a great amount of understanding to be gained from building circuits, seeing them work, and then understanding why they work. The objectives of the laboratories are as follows:

\begin{itemize}
	\item To verify concepts presented in the course.
	\item To enable students to gain skills in using basic electrical laboratory equipment such as meters, power supplies, and oscilloscopes, which will be need in subsequent electrical engineering laboratories and in industry.
	\item To make the students aware of the limits of the operation of components and equipment.
	\item To have the students gain skills in recording data and in reporting experimental results to effectively communicate what they have done and observed.
	\item To give students self-confidence in electrical laboratories.
	\item To have the students become safety conscious in a laboratory
environment.

\end{itemize}

%%%%%%%%%%%%%%%%%%%%%%%%%%%%%%%%%%%%%%%%%%%%%%%%%%%%%%%%%%%%%%%%%%%%%%%%%%%%%%%%%%%%%%%%%%%%%%%%%%%%%%%
\section{Requirements}
%%%%%%%%%%%%%%%%%%%%%%%%%%%%%%%%%%%%%%%%%%%%%%%%%%%%%%%%%%%%%%%%%%%%%%%%%%%%%%%%%%%%%%%%%%%%%%%%%%%%%%%

Each student is required to have the following:

\begin{itemize}
	\item A laptop with Waveforms installed
	\item If using a Macbook with the M1 chip (2020 or newer Macbooks) a parallel booting system is required
	\item Digilent Analog Discovery (1/2/NI)
	\item Breadboard
	\item Wiring kit
	\item Lab parts kit 
\end{itemize}

A Digilent Analog Discovery kit for this course can be purchased from the UF bookstore. It includes an Analog Discovery 2, Breadboard, wiring kit, and assorted parts.

For the wiring kit, I recommend this one. You want to use "solderless breadboard/ jumper" wires like at this link because the wires with other kits are more difficult to debug.
\url{https://www.amazon.com/REXQualis-Breadboard-Assorted-Prototyping-Circuits/dp/B081H2JQRV/ref=sr_1_1_sspa?dchild=1&keywords=solderless+breadboard+wires&qid=1610048019&sr=8-1-spons&psc=1&smid=AX8SR0V05IQ2E&spLa=ZW5jcnlwdGVkUXVhbGlmaWVyPUE4UEIyRlA1TllENDQmZW5jcnlwdGVkSWQ9QTAyNDM0MDkyVUoxNFEwTDBXM1YmZW5jcnlwdGVkQWRJZD1BMDMxMDkwMTFZSlUxMjhVTzBEQVkmd2lkZ2V0TmFtZT1zcF9hdGYmYWN0aW9uPWNsaWNrUmVkaXJlY3QmZG9Ob3RMb2dDbGljaz10cnVl}

\textbf{The lab manual is subject to change as errors are corrected, students are responsible for the newest version of the lab manual. }

%%%%%%%%%%%%%%%%%%%%%%%%%%%%%%%%%%%%%%%%%%%%%%%%%%%%%%%%%%%%%%%%%%%%%%%%%%%%%%%%%%%%%%%%%%%%%%%%%%%%%%%
\section{Structure}
%%%%%%%%%%%%%%%%%%%%%%%%%%%%%%%%%%%%%%%%%%%%%%%%%%%%%%%%%%%%%%%%%%%%%%%%%%%%%%%%%%%%%%%%%%%%%%%%%%%%%%%

The laboratory portion of the course is made up of nine labs and a final project and takes place in NEB 250. Every laboratory will be composed of an objective, list of materials, introduction, big picture, pre-lab requirements, in-lab requirements, and a write up. 


\begin{enumerate}
	\item \textbf{Objective}: States the goals for a given lab.
	\item \textbf{Materials}: List the materials used for a lab.
	\item \textbf{Introduction}: Introduces the overall theme of the lab and the supporting background information.
	\item \textbf{Big Picture}: A block diagram of the final project showing how the circuits in this lab fit together in the final project. 
	\item \textbf{Pre-Lab Requirements}: Most of the lab is accomplished by the student at home. The requirements for the pre-lab will list out what needs to be completed by students prior to their entry in to the lab.  
		\begin{itemize}
			\item Pre-lab quizzes are available on Canvas for 24 hours before the start of lab and must be completed by 8:30am on the day of your lab. 
			\item Pre-lab submissions to Canvas: tables, circuit diagrams and their resulting outputs, etc., must be compiled in to a document and submitted to Canvas by 8:30am on the day of your lab. No late submissions are accepted.
			\item Pre-lab circuit demonstrations must be ready at the start of lab.
			\item \textbf{Failure to complete the pre-lab on time, either the quiz, submission, or demonstration, with result in a zero for a given lab and the student will be barred from completing the lab.}
		\end{itemize}
	\item \textbf{In-Lab Requirements}: Additional lab tasks that build on the work done in the pre-lab. \textbf{Students arriving later than 5 minutes to lab will not be admitted.}
	\item \textbf{Write Up}: The conclusion of a lab requires a short write up, the project will require a formal lab report (more on that later) that summarizes the in-lab work. A typical write must have an introduction, summary of a results, discussion covering the work done in-lab (not the pre-lab), and a conclusion. See the template for details. Write ups due time will be posted on Canvas, typically due in one week, by 8:30am on the day of the of next lab.
\end{enumerate}

%
%\item \textbf{Pre-lab}: A one page document that must be completed before the laboratory. The pre-lab is due on the day the lab is performed and collected at the start of the class. Students who do not present a pre-lab will be denied access to that lab and receive a zero.
%\item \textbf{Background}: Details the theory associated with laboratory and contains useful equations and values.
%\item \textbf{Experiment}: List of tasks that must be completed in lab and a set of questions pertaining to the experiment that must be completed for the report
%\item \textbf{Lab Check}: The tasks which must be finished to complete a lab and shown to the lab instructor. 
%\item \textbf{Lab Report}: Each lab must be written up in to a report which includes an introduction, results, analysis, and conclusion. In general, don't include any of the figures or equations from the manual except the ones you make in the lab. 
	%\begin{enumerate}
			%\item \textbf{Abstract:} A brief summary of the experiment giving the purpose and explaining what is going to be shown or proved, and how this is to be done. The abstract should be fairly general. At lease four and no more than ten sentences. 
		%\item \textbf{Results}: List all the calculated and measured data in the form or tables or graphs, and calculate percent error when appropriate. 
		%\item \textbf{Discussion}: Answer all the questions in the lab in the appropriate order and interpret the results. The answers should be thorough and demonstrate understanding of the topics covered in the lab. Short or generic answers will receive no points.
		%\item \textbf{Conclusion}: Briefly conclude the lab by summarizing the results and discussion, at lease four and no more than ten sentences. 
	%\end{enumerate}
%\end{enumerate}

\noindent Consider the following when writing your lab write up.

\begin{itemize}
	\item For every value listed, include the appropriate units. 
	\item Be careful of the number of significant digits in the calculated results. This number should not be greater than the least number of significant digits in the data used for the calculations.
	\item Be objective and do not use colloquial phrases.
	\item Always give numerical results when applicable. Do not just say, ``the results
were as expected.'' Specify the actual results, the expected results, any errors, and explain them.
	\item Be concise. A short direct write up is more desirable than a long rambling one. But at the same time, don't give one word answers. The reader should clearly be able to tell that the writer understands the subject material. 
	\item Keep everything neat and label everything clearly.
	\item Remember to number the equations and use figure numbers and table numbers for any drawings or tables. Also include captions for all figures and tables.
	\item Percent error is defined as $\delta = |\frac{\mathrm{Theoretical - Experimental}}{\mathrm{Theoretical}}| \times \mathrm{100\%}$
	\item Use the file "Circuits 1 prelab and write up template"  in the "Lab related files" folder as a template for both Pre-Lab and Write-up reports.
	\item The write up must be typed, including tabled values, no pictures.
	\item All figures and tables must have a number and a caption.
	\item Include section labels, if it's unclear which section is which, you're more likely to lose points for not having a section.
	\item Don't make large unreadable tables and avoid weird table placement. All values must be appropriately tabled.
	\item Don't use large spacing, single spaced is fine.
	\item Don't include figures from the lab manual or equations unless they're asked for. You also don't need to show how you calculated voltage or percent error. Percent error is absolute value.
	\item Avoid rephrasing or copying the objective from the lab manual. The introduction and conclusion are worth a lot in the write up because the expectation is that you'll put time in to them. The is a cap of four sentences but that doesn't mean you can get away with one or two weak sentences.
	\item The discussion section should be a discussion on the results of lab and not discussing how you did the lab. What do the results show? You should aim for a few paragraphs.
\end{itemize}




%%%%%%%%%%%%%%%%%%%%%%%%%%%%%%%%%%%%%%%%%%%%%%%%%%%%%%%%%%%%%%%%%%%%%%%%%%%%%%%%%%%%%%%%%%%%%%%%%%%%%%%
\section{Tentative Schedule}
%%%%%%%%%%%%%%%%%%%%%%%%%%%%%%%%%%%%%%%%%%%%%%%%%%%%%%%%%%%%%%%%%%%%%%%%%%%%%%%%%%%%%%%%%%%%%%%%%%%%%%%

Subject to change. 

		\begin{itemize}
			\item Lab 1 - Week 3
			\item Lab 2 - Week 4
			\item Lab 3 - Week 5
			\item Lab 4 - Week 6
			\item Lab 5 - Week 7
			\item Lab 6 - Week 8
			\item Lab 7 - Week 9
			\item Lab 8 - Week 10
			\item Lab 9 - Week 11
			\item Lab Project - Week 12-13
		\end{itemize}


%%%%%%%%%%%%%%%%%%%%%%%%%%%%%%%%%%%%%%%%%%%%%%%%%%%%%%%%%%%%%%%%%%%%%%%%%%%%%%%%%%%%%%%%%%%%%%%%%%%%%%%
\section{Grading}
%%%%%%%%%%%%%%%%%%%%%%%%%%%%%%%%%%%%%%%%%%%%%%%%%%%%%%%%%%%%%%%%%%%%%%%%%%%%%%%%%%%%%%%%%%%%%%%%%%%%%%%

The grade for the lab section is made up labs and a project. Each lab is composed of a pre-lab, in-lab, and a write up. Students must complete the pre-lab, online submission and circuit demo, in order to complete the in-lab portion, failure to complete the pre-lab will result in a zero for the entire lab. 


\begin{itemize}
	\item Each lab is worth 10\% and the project is worth 20\% of the total lab grade. There is a total possible score of 110\% for the lab. A minimum score of 70\% in lab is required to pass the course, subject to change based on overall student performance. Completing the project is also required to pass the course.

	\item There is an on-line prelab quiz in Canvas required to be done, due at the same time with the prelab report (8:30am of the lab day).  It is a few basic concepts required for the lab and worth 10 points (except Lab 1).  Prelab report is worth 40 points, including the 10 points prelab demo circuits if required.  Students are required to attend the lab to complete their in-lab items to get 20 points.  Incomplete lab will get zero points for the In-lab.  No partial points.  In-lab results are required to be checked for completion.
Absence will result in zero points for both In-Lab (20 points)and Write-Up (30 points), because there is no proof of individual work performed.


	\item Each individual lab is graded as follows and will be accompanied by a rubric when appropriate:
		\begin{itemize}
			\item Pre-lab - 50\% = 10\% pre-lab quiz + 40\% pre-lab report (this includes the demo points)
			\item In-Lab - 20\%  You either get these points or your don't. There is no partial credit
			\item Write Up - 30\%  If absent, this is automatically a 0, since there is no proof of personal work
		\end{itemize}
\end{itemize}

%%%%%%%%%%%%%%%%%%%%%%%%%%%%%%%%%%%%%%%%%%%%%%%%%%%%%%%%%%%%%%%%%%%%%%%%%%%%%%%%%%%%%%%%%%%%%%%%%%%%%%%
\section{Makeup Policy}
%%%%%%%%%%%%%%%%%%%%%%%%%%%%%%%%%%%%%%%%%%%%%%%%%%%%%%%%%%%%%%%%%%%%%%%%%%%%%%%%%%%%%%%%%%%%%%%%%%%%%%%

Students are only allowed to make up missed labs with a legitimate absence and must produce documentation for their absence.

%%%%%%%%%%%%%%%%%%%%%%%%%%%%%%%%%%%%%%%%%%%%%%%%%%%%%%%%%%%%%%%%%%%%%%%%%%%%%%%%%%%%%%%%%%%%%%%%%%%%%%%
\section{Copying}
%%%%%%%%%%%%%%%%%%%%%%%%%%%%%%%%%%%%%%%%%%%%%%%%%%%%%%%%%%%%%%%%%%%%%%%%%%%%%%%%%%%%%%%%%%%%%%%%%%%%%%%

Students are bound by the University of Florida honor code and every student is expected to produce their own work. Any students who are found copying will receive a zero for the lab in question.

%%%%%%%%%%%%%%%%%%%%%%%%%%%%%%%%%%%%%%%%%%%%%%%%%%%%%%%%%%%%%%%%%%%%%%%%%%%%%%%%%%%%%%%%%%%%%%%%%%%%%%%
\section{Safety}
%%%%%%%%%%%%%%%%%%%%%%%%%%%%%%%%%%%%%%%%%%%%%%%%%%%%%%%%%%%%%%%%%%%%%%%%%%%%%%%%%%%%%%%%%%%%%%%%%%%%%%%

Although unlikely, it is possible to become injured in this laboratory. Emergency contacts by phone are 911 and 395-0050, the latter is for the Emergency Room at Shands Hospital. On some university phones it may be necessary to dial 9 first.

When working in any electrical laboratory, always keep electrical safety in mind. Following are some safety rules that each student should be aware of before beginning a laboratory experiment. Some of the rules do not apply to this laboratory, but to other electrical laboratories.

\begin{enumerate}
	\item Never work totally alone in the laboratory. Someone else, and preferably the laboratory TA, should be present in the case of an emergency.
	\item Use only the equipment provided. Do not use other equipment unless the laboratory TA approves the use.
	\item Turn off the power before handling any wires. Never use damaged items, whether they are leads, components, equipment, or any other item.
	\item To decrease the chances of being shocked, wear dry shoes and do not stand on metal or wet concrete. Also, do not wear any metal or jewelry. Moreover, do not handle wires, components,
or equipment with wet hands.
	\item In a laboratory in which soldering irons are used, keep an attentive eye on a hot soldering iron. Also, place it in the proper holder when not in use. Never leave a hot soldering iron
unattended.
	\item Make no connection to the power supply until the very last step. This practice will ensure that a student who is handling leads will not be shocked. Also, the circuit being built will not be harmed during construction.
	\item Should someone become incapacitated because of electric shock, no one should touch that person until the power is turned off. Otherwise, the rescuer could also be shocked. Call for emergency services as soon as possible since resuscitation is likely if treatment is applied quickly. If breathing has stopped, begin CPR immediately and continue until qualified medical assistance has arrived.
	\item If a piece of equipment is not working properly, attach a note and report the problem to the lab TA. The malfunction could possibly lead to a life-threatening situation.
\end{enumerate}
